\documentclass[withoutpreface]{cumcmthesis}
\title{基于logisim的单周期硬布线24指令MIPS-CPU设计}
\usepackage{graphicx} 
\begin{document}
\maketitle
\begin{tabular}{cccccc}
	\hline
	学  号 & E12214052 &专  业 & 计算机科学与技术 &姓  名 & 赵宸宇 \\
	\hline
	实验日期 & 2024年10月10日 &教师签字 &  &成  绩&\\
	\hline
\end{tabular}
\begin{abstract}
	通过本次实验,我完成了以下任务,实现了以下实验目的:
	\subsection*{实验的主要目的:}
	
	在Logisim中实现能运行\textbf{24条基础指令},能运行\textbf{标准测试程序}的Mips单周期硬布线CPU。
	
	\subsection*{实验的主要任务:}
	
	\begin{enumerate}
		\item 构建MIPSCPU\textbf{数据通路}
		\item \textbf{单周期硬布线控制器}实现
		\item 软硬件\textbf{测试联调}

	\end{enumerate}
	
	

\subsection*{实验产出:}
\begin{enumerate}
	\item 自主设计电路图
	\item 头哥网通关
	\item 实验报告(\LaTeX{})
	\item 支持材料
	\item \textbf{gitee}仓库增量更新 请见\url{https://gitee.com/cslearnerer/AHU-CSHT}
\end{enumerate}
\end{abstract}
\tableofcontents
\newpage

\section{【实验内容】}
\subsection{构建MIPSCPU数据通路}
首先,根据数据通路原理图,使用给定部件构建数据通路。这里选用原版reg file,以方便后续实验观察运行情况。
\begin{figure}[!h]
	\centering
	\includegraphics[width=0.7\linewidth]{"../../../../../../iobsidian/AhuEx/■笔记/46 系统硬件综合训练 徐晨初/-attachments/Pasted image 20241010143111"}
	\caption{给定的数据通路蓝图}
	\label{fig:pasted-image-20241010143111}
\end{figure}


\subsection{单周期硬布线控制器实现}

\subsection{软硬件测试联调}


\section{【小结讨论】}


\end{document}